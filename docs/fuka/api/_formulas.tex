\documentclass{article}
\usepackage{ifthen}
\usepackage{epsfig}
\usepackage[utf8]{inputenc}
\usepackage{newunicodechar}
  \newunicodechar{⁻}{${}^{-}$}% Superscript minus
  \newunicodechar{²}{${}^{2}$}% Superscript two
  \newunicodechar{³}{${}^{3}$}% Superscript three

\pagestyle{empty}
\begin{document}
$\varphi$
\pagebreak

$ z=0 $
\pagebreak

$(X_c, Z_c)$
\pagebreak

$ \rho = (R - r_i(\theta))/2. x + (R + r_i(\theta))/2. $
\pagebreak

$ \theta = \theta^\star $
\pagebreak

$ R $
\pagebreak

$ r_i $
\pagebreak

$ r = (r_o(\theta) - R)/2. x + (r_o(\theta) + R)/2. $
\pagebreak

$ r_o $
\pagebreak

$r_i$
\pagebreak

$ R$
\pagebreak

$ {\rm d} r / {\rm d} x$
\pagebreak

$ {\rm d} r / {\rm d} \theta$
\pagebreak

$ \partial_r $
\pagebreak

$ \partial_\theta $
\pagebreak

$(X_c, Y_c, Z_c)$
\pagebreak

$ r = (R - r_i(\theta, \varphi))/2. x + (R + r_i(\theta, \varphi))/2. $
\pagebreak

$ \varphi = \varphi^\star $
\pagebreak

$ r = (r_o(\theta, \varphi) - R)/2. x + (r_o(\theta, \varphi) + R)/2. $
\pagebreak

$ {\rm d} r / {\rm d} \varphi$
\pagebreak

$r_o$
\pagebreak

$ d $
\pagebreak

$ x_1 $
\pagebreak

$ x_2 $
\pagebreak

$ R $
\pagebreak

$ \partial_\varphi $
\pagebreak

$ r_1 $
\pagebreak

$ r_2 $
\pagebreak

$ r_1/2 $
\pagebreak

$ r_1/2 < r < r_1 $
\pagebreak

$ r_1 < r < 2 r_1 $
\pagebreak

$ r_2/2 $
\pagebreak

$ r_2/2 < r < r_2 $
\pagebreak

$ r_2 < r < 2 r_2 $
\pagebreak

$ BH1_bounds(0) $
\pagebreak

$ BH1_bounds(0) < r < BH1_bounds(1) $
\pagebreak

$ BH1_bounds(1) < r < BH1_bounds(2) $
\pagebreak

$ BH1_bounds(2) < r < BH1_bounds(3+n_shells1) $
\pagebreak

$ BH2_bounds(0) $
\pagebreak

$ BH2_bounds(0) < r < BH2_bounds(1) $
\pagebreak

$ BH2_bounds(1) < r < BH2_bounds(2) $
\pagebreak

$ BH2_bounds(2) < r < BH2_bounds(3+n_shells2) $
\pagebreak

$ outer_bounds(len-1) $
\pagebreak

$\theta$
\pagebreak

$ nshells1 $
\pagebreak

$ nshells2 $
\pagebreak

$ <= nz $
\pagebreak

$ \chi $
\pagebreak

$ \eta $
\pagebreak

$ -1 \leq \eta^\star \leq 1 $
\pagebreak

$ 0 \leq \chi^\star \leq 1 $
\pagebreak

$ 0 \leq \varphi^\star < 2\pi $
\pagebreak

$ \eta = \displaystyle\frac{\eta_{\rm max}-\eta_{\rm min}}{2} \eta^\star + \displaystyle\frac{\eta_{\rm max}+\eta_{\rm min}}{2} $
\pagebreak

$ \chi = (\chi_{\rm min}- \pi) \chi^\star + \pi $
\pagebreak

$ x = a \displaystyle\frac{\sinh \eta}{\cosh\eta - \cos\chi} $
\pagebreak

$ y = a \displaystyle\frac{\cosh \eta \cos\varphi}{\cosh\eta - \cos\chi} $
\pagebreak

$ z = a \displaystyle\frac{\cosh \eta \sin\varphi}{\cosh\eta - \cos\chi} $
\pagebreak

$ \eta = \displaystyle\frac{f(\chi)-\eta_{\rm lim}}{2} \eta^\star + \displaystyle\frac{f(\chi)+\eta_{\rm lim}}{2} $
\pagebreak

$ \chi = \chi_{\rm max} \chi^\star $
\pagebreak

$ f (\chi) $
\pagebreak

$ f $
\pagebreak

$ \displaystyle\frac{\sinh ^2 f(\chi) + \sin ^2\chi} {(\cosh f(\chi) - \cos \chi)^2} = \displaystyle\frac{R^2}{a^2} $
\pagebreak

$ \chi = (g(\eta)-\pi) \chi^\star + \pi $
\pagebreak

$ g (\eta) $
\pagebreak

$ g $
\pagebreak

$ \displaystyle\frac{\sinh ^2 \eta + \sin ^2 g(\eta)} {(\cosh \eta - \cos g(\eta))^2} = \displaystyle\frac{R^2}{a^2} $
\pagebreak

$ a $
\pagebreak

$ x=-1 $
\pagebreak

$ x=+1 $
\pagebreak

$ \varphi $
\pagebreak

$ \displaystyle\frac{\partial \eta^\star}{\partial x} $
\pagebreak

$ \displaystyle\frac{\partial \eta}{\partial x} = \displaystyle\frac{2 a (a^2 + r^2 - 2x^2)}{(a^2+r^2)^2 - 4a^2x^2} $
\pagebreak

$ \displaystyle\frac{\partial \eta^\star}{\partial \eta } = \displaystyle\frac{2}{\eta_{\rm max} - \eta_{\rm min}} $
\pagebreak

$ \displaystyle\frac{\partial \eta^\star}{\partial y} $
\pagebreak

$ \displaystyle\frac{\partial \eta}{\partial y} = \displaystyle\frac{-4axy}{(a^2+r^2)^2 - 4a^2x^2} $
\pagebreak

$ \displaystyle\frac{\partial \eta^\star}{\partial z} $
\pagebreak

$ \displaystyle\frac{\partial \eta}{\partial z} = \displaystyle\frac{-4axz}{(a^2+r^2)^2 - 4a^2x^2} $
\pagebreak

$ \displaystyle\frac{\partial \chi^\star}{\partial x} $
\pagebreak

$ \displaystyle\frac{\partial \chi}{\partial x} = \displaystyle\frac{-4ax\sqrt{y^2+z^2}}{(r^2-a^2)^2 + 4 a^2 (y^2+z^2)} $
\pagebreak

$ \displaystyle\frac{\partial \chi^\star}{\partial \chi } = \displaystyle\frac{1}{\chi_{\rm min} - \pi} $
\pagebreak

$ \displaystyle\frac{\partial \chi^\star}{\partial y} $
\pagebreak

$ \displaystyle\frac{\partial \chi}{\partial y} = \cos\varphi \displaystyle\frac{2ax(r^2-a^2-a^2(y^2+z^2)}{(r^2-a^2)^2 + 4 a^2 (y^2+z^2)} $
\pagebreak

$ \displaystyle\frac{\partial \chi}{\partial z} = \sin\varphi \displaystyle\frac{2ax(r^2-a^2-a^2(y^2+z^2)}{(r^2-a^2)^2 + 4 a^2 (y^2+z^2)} $
\pagebreak

$ \sin\chi \displaystyle\frac{\partial \varphi^\star}{\partial y} $
\pagebreak

$ \sin\chi \displaystyle\frac{\partial \varphi^\star}{\partial y} = - \displaystyle\frac{(\cosh\eta - \cos\chi) \sin\varphi}{a} $
\pagebreak

$ \sin\chi \displaystyle\frac{\partial \varphi^\star}{\partial z} $
\pagebreak

$ \sin\chi \displaystyle\frac{\partial \varphi^\star}{\partial z} = \displaystyle\frac{(\cosh\eta - \cos\chi) \cos\varphi}{a} $
\pagebreak

$ \eta_{\rm min} $
\pagebreak

$ x = -1 $
\pagebreak

$ \eta_{\rm max} $
\pagebreak

$ x = +1 $
\pagebreak

$ \chi_{\rm min} $
\pagebreak

$ \displaystyle\frac{\partial \eta^\star}{\partial X} $
\pagebreak

$ cos(m\varphi^\star)$
\pagebreak

$ T_(2j) (\chi^\star)$
\pagebreak

$ m $
\pagebreak

$ T_(2j+1) (\chi^\star)$
\pagebreak

$ T_{i} (\eta^\star) $
\pagebreak

$ P_(2j) (\chi^\star)$
\pagebreak

$ P_(2j+1) (\chi^\star)$
\pagebreak

$ P_{i} (\eta^\star) $
\pagebreak

$ sin(m\varphi^\star)$
\pagebreak

$ \eta^\star = \displaystyle\frac{2}{\eta_{\rm max}-\eta_{\rm min}} ({\rm atanh}(\displaystyle\frac{2ax}{a^2+r^2}) - \displaystyle\frac{\eta_{\rm max}+\eta_{\rm min}}{2}) $
\pagebreak

$ \chi^\star = \displaystyle\frac{({\rm atan} (\displaystyle\frac{2a\sqrt{y^2+z^2}}{r^2-a^2} - \pi)} {\chi_{\rm min} - \pi} $
\pagebreak

$ \varphi^\star = {\rm atan} \displaystyle\frac{z}{y} $
\pagebreak

$ (x, y, z) $
\pagebreak

$ (x, \theta^\star, \varphi^\star) $
\pagebreak

$ 1/r^2$
\pagebreak

$ f (\chi=0) = \ln (\frac{R}{a}+1) - \ln (\frac{R}{a}-1) $
\pagebreak

$ \chi^\star $
\pagebreak

$ \displaystyle\frac{\partial \eta^\star}{\partial x } = (\displaystyle\frac{\partial \eta}{\partial x} - \displaystyle\frac{\partial f}{2\partial x}) (\displaystyle\frac{2}{f - \eta_{\rm lim}}) + (\eta - \displaystyle\frac{f+\eta_{\rm lim}}{2}) (-\displaystyle\frac{2}{(f-\eta_{\rm lim})^2} \displaystyle\frac{\partial f}{\partial x}) $
\pagebreak

$ \displaystyle\frac{\partial f}{\partial x} = f'(\chi) \displaystyle\frac{\partial \chi}{\partial x}$
\pagebreak

$ \displaystyle\frac{\partial \eta^\star}{\partial y } = (\displaystyle\frac{\partial \eta}{\partial y} - \displaystyle\frac{\partial f}{2\partial y}) (\displaystyle\frac{2}{f - \eta_{\rm lim}}) + (\eta - \displaystyle\frac{f+\eta_{\rm lim}}{2}) (-\displaystyle\frac{2}{(f-\eta_{\rm lim})^2} \displaystyle\frac{\partial f}{\partial y}) $
\pagebreak

$ \displaystyle\frac{\partial f}{\partial y} = f'(\chi) \displaystyle\frac{\partial \chi}{\partial y}$
\pagebreak

$ \displaystyle\frac{\partial \eta^\star}{\partial z } = (\displaystyle\frac{\partial \eta}{\partial z} - \displaystyle\frac{\partial f}{2\partial z}) (\displaystyle\frac{2}{f - \eta_{\rm lim}}) + (\eta - \displaystyle\frac{f+\eta_{\rm lim}}{2}) (-\displaystyle\frac{2}{(f-\eta_{\rm lim})^2} \displaystyle\frac{\partial f}{\partial z}) $
\pagebreak

$ \displaystyle\frac{\partial f}{\partial z} = f'(\chi) \displaystyle\frac{\partial \chi}{\partial z}$
\pagebreak

$ \displaystyle\frac{\partial \chi^\star}{\partial \chi } = \displaystyle\frac{1}{\chi_{\rm max}} $
\pagebreak

$ \displaystyle\frac{\partial \chi^\star}{\partial z} $
\pagebreak

$ \eta_{\rm lim} $
\pagebreak

$ \chi_{\rm max} $
\pagebreak

$ f(\chi) $
\pagebreak

$ *p_bound_eta $
\pagebreak

$ *_bound_eta_der $
\pagebreak

$ \eta^\star = \displaystyle\frac{2}{f(\chi)-\eta_{\rm lim}} ({\rm atanh}(\displaystyle\frac{2ax}{a^2+r^2}) - \displaystyle\frac{f(\chi)+\eta_{\rm lim}}{2}) $
\pagebreak

$ \chi^\star = \displaystyle\frac{({\rm atan} \displaystyle\frac{2a\sqrt{y^2+z^2}}{r^2-a^2}} {\chi_{\rm max}} $
\pagebreak

$ 0 \leq X \leq 1 $
\pagebreak

$ 0 \leq T^\star \leq \pi $
\pagebreak

$ X = x_{\rm lim} X^\star $
\pagebreak

$ T = T^\star $
\pagebreak

$ -1 \leq X \leq 1 $
\pagebreak

$ X = \left(1 - x_{\rm lim}\right)/2 X^\star + \left(1 - x_{\rm lim}\right)/2 $
\pagebreak

$ x_{\rm lim}$
\pagebreak

$ X$
\pagebreak

$ T$
\pagebreak

$ X^\star $
\pagebreak

$ x_{\rm lim} $
\pagebreak

$ g (\eta=0) = 2*{\rm atan}(\displaystyle\frac{a}{R}) $
\pagebreak

$ \eta^\star $
\pagebreak

$ \displaystyle\frac{\partial \chi^\star}{\partial x } = \displaystyle\frac{\displaystyle\frac{\partial \chi}{\partial x}(g(\eta) - \pi) - \displaystyle\frac{\partial g}{\partial x}(\chi - \pi)}{(g(\eta)-\pi)^2} $
\pagebreak

$ \displaystyle\frac{\partial g}{\partial x} = g'(\eta) \displaystyle\frac{\partial \eta}{\partial x}$
\pagebreak

$ \displaystyle\frac{\partial \chi^\star}{\partial y} = \displaystyle\frac{\displaystyle\frac{\partial \chi}{\partial y}(g(\eta) - \pi) - \displaystyle\frac{\partial g}{\partial y}(\chi - \pi)}{(g(\eta)-\pi)^2} $
\pagebreak

$ \displaystyle\frac{\partial g}{\partial y} = g'(\eta) \displaystyle\frac{\partial \eta}{\partial y}$
\pagebreak

$ \displaystyle\frac{\partial \chi^\star}{\partial z} = \displaystyle\frac{\displaystyle\frac{\partial \chi}{\partial z}(g(\eta) - \pi) - \displaystyle\frac{\partial g}{\partial z}(\chi - \pi)}{(g(\eta)-\pi)^2} $
\pagebreak

$ \displaystyle\frac{\partial g}{\partial z} = g'(\eta) \displaystyle\frac{\partial \eta}{\partial z}$
\pagebreak

$ g(\eta) $
\pagebreak

$ *p_bound_chi $
\pagebreak

$ *_bound_chi_der $
\pagebreak

$ \chi^\star = \displaystyle\frac{({\rm atan} \displaystyle\frac{2a\sqrt{y^2+z^2}}{r^2-a^2}-\pi)} {(g(\eta) - \pi)} $
\pagebreak

$ X=0$
\pagebreak

$ X=1 $
\pagebreak

$(r,\theta, \varphi)$
\pagebreak

$\pi$
\pagebreak

$ 0 \leq x \leq 1 $
\pagebreak

$ 0 \leq \theta^\star \leq \pi/2 $
\pagebreak

$ 0 \leq t^\star \leq \quad \pi$
\pagebreak

$ r = \alpha x $
\pagebreak

$ t = t^\star / \omega$
\pagebreak

$T$
\pagebreak

$ -1 \leq x \leq 1 $
\pagebreak

$ r = \alpha x + \beta$
\pagebreak

$ t^\star$
\pagebreak

$ \pi$
\pagebreak

$\pi/2$
\pagebreak

$ nr-1 $
\pagebreak

$ \sqrt{a^2 + r_1^2} + \sqrt{a^2 + r_2^2} = d $
\pagebreak

$ \eta_{\rm minus} = -{\rm asinh}\displaystyle\frac{a}{r_1}$
\pagebreak

$ x_1 = a \displaystyle\frac{\cosh \eta_{\rm minus}}{\sinh \eta_{\rm minus}} $
\pagebreak

$ \eta_{\rm plus} = {\rm asinh}\displaystyle\frac{a}{r_2}$
\pagebreak

$ x_2 = a \displaystyle\frac{\cosh \eta_{\rm plus}}{\sinh \eta_{\rm plus}} $
\pagebreak

$ \chi_c = 2 {\rm atan}\displaystyle\frac{a}{R} $
\pagebreak

$ \eta_c = \ln (\displaystyle\frac{R}{a} + 1) - \ln (\displaystyle\frac{R}{a} - 1) $
\pagebreak

$ 1/r$
\pagebreak

$ r = (R - r_i)/2. x + (R + r_i)/2. $
\pagebreak

$ r = (r_o - R)/2. x + (r_o + R)/2. $
\pagebreak

$ A(i, j)$
\pagebreak

$[ku+1+i-j, j]$
\pagebreak

$\mathrm {max}(1, j-ku) \leq i \leq \mathrm{min} (n, j+kl)$
\pagebreak

$varphi$
\pagebreak

$ (i,j) $
\pagebreak

$\gamma_{ij} = \frac{1}{\Omega^2}\tilde{\gamma}_{ij}$
\pagebreak

$ \partial_r, \frac{1}{r}\partial_\theta, 0$
\pagebreak

$\partial_i$
\pagebreak

$\partial^i$
\pagebreak

$D_i$
\pagebreak

$D^i$
\pagebreak

$\bar{D}_i$
\pagebreak

$\bar{D}^i$
\pagebreak

$ \partial_r, \frac{1}{r}\partial_\theta, \frac{1}{r\sin\theta}\partial_\varphi$
\pagebreak

$ \partial_r, \frac{\cos \theta}{r}\partial_\theta, \frac{\cos \theta}{r\sin\theta}\partial_\varphi$
\pagebreak

$ r = \alpha x + \beta $
\pagebreak

$ r = \frac{1}{\left(\alpha x -1\right)} $
\pagebreak

$ \frac{r^2}{L^2} $
\pagebreak

$\epsilon = (1 - \frac{r^2}{L^2})/(1 + \frac{r^2}{L^2})$
\pagebreak

$\epsilon^n$
\pagebreak

$h_{ij}$
\pagebreak

$ D_i \epsilon$
\pagebreak

$r$
\pagebreak

$x$
\pagebreak

$\frac{1}{r} \partial_r$
\pagebreak

$ h_{ij} $
\pagebreak

$ \alpha $
\pagebreak

$(r,\theta)$
\pagebreak

$ \rho = \alpha x $
\pagebreak

$ \rho = \alpha x + \beta $
\pagebreak

$ \rho = \displaystyle\frac{1}{\alpha x -1}$
\pagebreak

$ r$
\pagebreak

$ r\sin \theta$
\pagebreak

$ 1-x^2$
\pagebreak

$\cos \theta$
\pagebreak

$\sin \theta$
\pagebreak

$\cos \varphi$
\pagebreak

$\sin \varphi$
\pagebreak

$\alpha$
\pagebreak

$(r,\varphi)$
\pagebreak

$(\theta, \varphi)$
\pagebreak

$X$
\pagebreak

$ \varphi$
\pagebreak

$ 0 \leq t^\star \leq \pi/2 \quad {\rm or} \quad \pi$
\pagebreak

$ r = \frac{1}{\alpha \left(x-1\right)} $
\pagebreak

$ X = r \sin\theta \cos\varphi + Xc $
\pagebreak

$ Y = r \sin\theta \sin\varphi + Yc $
\pagebreak

$ Z = r \cos\theta + Zc $
\pagebreak

$ \pi/2$
\pagebreak

$\varphi $
\pagebreak

$ 0 \leq \varphi^\star < \pi /2 $
\pagebreak

$ 0 \leq \varphi^\star \leq \pi/2 $
\pagebreak

$ r = \displaystyle\frac{1}{\alpha x -1}$
\pagebreak

$ 0 \leq \varphi^\star \leq /pi /2 $
\pagebreak

$ \log r = \alpha x + \beta $
\pagebreak

$ 1/r $
\pagebreak

$ X $
\pagebreak

$ T $
\pagebreak

$ 0 \leq i < $
\pagebreak

$ cos((2j)\theta^\star)$
\pagebreak

$ l = 2j $
\pagebreak

$ sin((2j+1) \theta^\star)$
\pagebreak

$ l = 2j+1 $
\pagebreak

$ T_{2i} (x)$
\pagebreak

$ l $
\pagebreak

$ T_{2i+1} (x)$
\pagebreak

$ P_{2i} (x)$
\pagebreak

$ P_{2i+1} (x)$
\pagebreak

$ cos((2j+1)\theta^\star)$
\pagebreak

$ sin((2j) \theta^\star)$
\pagebreak

$z$
\pagebreak

$y$
\pagebreak

$ m$
\pagebreak

$ \theta$
\pagebreak

$ \varphi$
\pagebreak

$ x = \displaystyle\frac{\sqrt{(X-X_c)^2+(Y-Y_c)^2+(Z-Z_c)^2}}{\alpha} $
\pagebreak

$ \theta^\star = {\rm atan} (\displaystyle\frac{\sqrt{(X-X_c)^2+(Y-Y_c)^2}}{Z-Z_c}) $
\pagebreak

$ \varphi^\star = {\rm atan} (\displaystyle\frac{Y-Y_c}{Z-Z_c}) $
\pagebreak

$ (X, Y, Z) $
\pagebreak

$(r, \theta)$
\pagebreak

$(r, \varphi)$
\pagebreak

$ \displaystyle\frac{\partial f}{\partial X} = (\sin\theta \displaystyle\frac{\partial f}{\alpha\partial x} + \cos\theta \displaystyle\frac{\partial f}{r\partial \theta^\star}) \cos\varphi - \sin\varphi \displaystyle\frac{\partial f}{r\sin\theta \partial \varphi^\star} $
\pagebreak

$ \displaystyle\frac{\partial f}{\partial Y} = (\sin\theta \displaystyle\frac{\partial f}{\alpha\partial x} + \cos\theta \displaystyle\frac{\partial f}{r\partial \theta^\star}) \sin\varphi + \cos\varphi \displaystyle\frac{\partial f}{r\sin\theta \partial \varphi^\star} $
\pagebreak

$ \displaystyle\frac{\partial f}{\partial Z} = \cos\theta \displaystyle\frac{\partial f}{\alpha\partial x} - \sin\theta\displaystyle\frac{\partial f}{r\partial \theta^\star}$
\pagebreak

$ x$
\pagebreak

$ y$
\pagebreak

$ z$
\pagebreak

$ (x,y) $
\pagebreak

$ (x,z) $
\pagebreak

$ (y, z) $
\pagebreak

$1/r$
\pagebreak

$ \cos \varphi$
\pagebreak

$ \sin \varphi$
\pagebreak

$ \cos \theta$
\pagebreak

$ \sin \theta$
\pagebreak

$ x$
\pagebreak

$ \chi$
\pagebreak

$ (x-1)$
\pagebreak

$ (1-x^2)$
\pagebreak

$ \alpha$
\pagebreak

$ (x+1)$
\pagebreak

$ \sin \chi$
\pagebreak

$ \cos \omega t$
\pagebreak

$ \sin \omega t$
\pagebreak

$ \theta$
\pagebreak

$ r^2$
\pagebreak

$ f' / r$
\pagebreak

$ f$
\pagebreak

$ T_{i} (x)$
\pagebreak

$ P_{i} (x)$
\pagebreak

$1-x^2$
\pagebreak

$ x = \displaystyle\frac{\sqrt{(X-X_c)^2+(Y-Y_c)^2+(Z-Z_c)^2}-\beta}{\alpha} $
\pagebreak

$ -1 \leq t^\star \leq 1 $
\pagebreak

$ t = \left (t_{\rm max} + t_{\rm min} \right)/2 \times t^\star + \left (t_{\rm max} - t_{\rm min} \right)/2$
\pagebreak

$ t_{\rm min} $
\pagebreak

$ t_{\rm max} $
\pagebreak

$ 1 - r/L$
\pagebreak

$(\partial_r, \partial_\theta /r , \partial_\varphi/r/sin\theta)$
\pagebreak

$(\partial_r, \frac{\cos \theta}{r} \partial_\theta , \frac{\cos \theta}{r \sin \theta} \partial_\varphi)$
\pagebreak

$ x = 1 + \displaystyle\frac{1}{\alpha\sqrt{(X-X_c)^2+(Y-Y_c)^2+(Z-Z_c)^2}} $
\pagebreak

$ r\displaystyle\frac{\partial f}{\partial X} = (-\sin\theta (x-1) \displaystyle\frac{\partial f}{\partial x} + \cos\theta \displaystyle\frac{\partial f}{r\partial \theta^\star}) \cos\varphi - \sin\varphi \displaystyle\frac{\partial f}{r\sin\theta \partial \varphi^\star} $
\pagebreak

$ r\displaystyle\frac{\partial f}{\partial Y} = (-\sin\theta (x-1) \displaystyle\frac{\partial f}{\partial x} + \cos\theta \displaystyle\frac{\partial f}{r\partial \theta^\star}) \sin\varphi + \cos\varphi \displaystyle\frac{\partial f}{r\sin\theta \partial \varphi^\star} $
\pagebreak

$ r\displaystyle\frac{\partial f}{\partial Z} = -\cos\theta (x-1) \displaystyle\frac{\partial f}{\partial x} - \sin\theta\displaystyle\frac{\partial f}{r\partial \theta^\star}$
\pagebreak

$m_{\rm order}$
\pagebreak

$m_{\rm quant}$
\pagebreak

$m$
\pagebreak

$ \sin \theta $
\pagebreak

$ \cos \theta $
\pagebreak

$ \sin \varphi $
\pagebreak

$ \cos \varphi $
\pagebreak

$l=m=0$
\pagebreak

$ x $
\pagebreak

$ x-1 $
\pagebreak

$ 1-x^2 $
\pagebreak

$ \sin \chi $
\pagebreak

$r^2$
\pagebreak

$\cos \omega t$
\pagebreak

$\sin \omega t$
\pagebreak

$ J \times x$
\pagebreak

$J$
\pagebreak

$\delta$
\pagebreak

\end{document}
